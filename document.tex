\documentclass[11pt,xcolor=dvipsnames]{beamer}
\usetheme{Amsterdam}
%\usecolortheme{rose}
%%%%Load the necessary packages %%%%
\usepackage[T1]{fontenc}
\usepackage{graphicx}
\usepackage{epigraph}
\usepackage{booktabs}
\usepackage{tikz}
\usepackage{lmodern}
\usepackage{siunitx}
\sisetup{table-number-alignment=right}
\usepackage[style=numeric,backend=biber,doi=false,isbn=false,sorting=ynt]{biblatex}
\addbibresource{rickets.bib}
%%%%%Set color and document specific stuff %%%%%%%%%%%%%%%%%%%%%%%
\setbeamertemplate{frametitle}{\textsc{\insertframetitle}}
\setbeamercolor{bibliography entry note}{fg=black}
\newenvironment<>{tealbox}[1]{%
  \setbeamercolor{block title}{fg=white,bg=teal!75!black}%
  \begin{block}#2{#1}}{\end{block}}
  \newenvironment<>{redbox}[1]{%
  \setbeamercolor{block title}{fg=white,bg=red!75!black}%
  \begin{block}#2{#1}}{\end{block}}
  
%%%%%%%%%%%%%%%%%%%%%%%%%%%%%%%%%%%%%%

%%%%%%%%%%%%%%%%%%%%%%%%%%%%%%%%%%%%%%

\author{Karthik}
\title{Testing STATA with LaTeX}
%\subtitle{}
%\logo{}
\institute{Jawaharlal Institute of Postgraduate Medical Education and Research}
%\date{}
%\subject{}
%\setbeamercovered{transparent}
\setbeamertemplate{navigation symbols}{}

\begin{document}
	\maketitle
\section{Introduction}	
	\begin{frame}
		\frametitle{This is how we test the work}
		our world works quite well for all the stuff we know to be true.
	\end{frame}
\section{Methods}
\begin{frame}
	\frametitle{However we do}
	Our ability to do something like this is quite limited by the fact that our models are clear and easy to understand.
\end{frame}
%%%%%%%%%%%%%%%%%%%%%%%%%%%%%%%%%%%%%%

\begin{frame}
	\begin{center}
		\epigraph{This seems to be a great idea}{\textit{Karthik Balachandran}}
	\end{center}
\end{frame}
%%%%%%%%%%%%%%%%%%%%%%%%%%%%%%%%%%%%%%
\begin{frame}
	\begin{theorem}
		Testing a theorem
	\end{theorem}
\end{frame}
%%%%%%%%%%%%%%%%%%%%%%%%%%%%%%%%%%%%%%
%%%%%%%%%%%%%%%%%%%%%%%%%%%%%%%%%%%%%%%%%%%%%%%%%%%%%%%%%%%%%%%
\begin{frame}
This is how we ensured that the work we do is real.	\cite{Allgrove} We also know how this person worked his way to glory.\cite{Glorieux.2014}
\end{frame}
%%%%%%%%%%%%%%%%%%%%%%%%%%%%%%%%%%%%%%%%%%%%%%%%%%%%%%%%%%%%%%%
\begin{frame}{Frame Title}
	This is also a very useful command if you are going to make sure it works. \footfullcite{Linglart.2014}
\end{frame}
%%%%%%%%%%%%%%%%%%%%%%%%%%%%%%%%%%%%%%%%%%%%%%%%%%%%%%%%%%%%%%%
\begin{frame}{An idea of how Knalij works}
\begin{figure}[htbp]
		\centering
		\includegraphics[width=\textwidth]{../../Karthik/Pictures/knalij.PNG}
		\label{fig:knalij}
	\end{figure}	
\end{frame}
%%%%%%%%%%%%%%%%%%%%%%%%%%%%%%%%%%%%%%%%%%%%%%%%%%%%%%%%%%%%%%%
\begin{frame}
	\begin{redbox}{Good for nothin}
	Should we play a lot of working thing? I don't know.
	\end{redbox}
\end{frame}
%%%%%%%%%%%%%%%%%%%%%%%%%%%%%%%%%%%%%%%%%%%%%%%%%%%%%%%%%%%%%%%
\begin{frame}
\begin{table}[htbp]
		\centering
		\normalsize
		\begin{tabular}{ll}
			\toprule
				One & Two\\
			\midrule
				Three & Four\\
			
				Five & Six\\
			\bottomrule
		\end{tabular}
		\caption{Testing table}
	\end{table}
		
\end{frame}

%%%%%%%%%%%%%%%%%%%%%%%%%%%%%%%%%%%%%%%%%%%%%%%%%%%%%%%%%%%%%%%
\begin{frame}
\begin{itemize}
		\item One
		\item Two
		\item Three
		\item Four
\end{itemize}
	
\end{frame}

%%%%%%%%%%%%%%%%%%%%%%%%%%%%%%%%%%%%%%%%%%%%%%%%%%%%%%%%%%%%%%%
\begin{frame}[allowframebreaks]	
			\printbibliography	
\end{frame}

\end{document}